\documentclass[conference]{IEEEtran}
\IEEEoverridecommandlockouts
% The preceding line is only needed to identify funding in the first footnote. If that is unneeded, please comment it out.
\usepackage{url}
%\usepackage{hyperref}
\usepackage{cite}
\usepackage{amsmath,amssymb,amsfonts}
\usepackage{algorithmic}
\usepackage{graphicx}
\usepackage{textcomp}
\usepackage{xcolor}
\usepackage{todo}
\usepackage{tikz}
\usepackage{subfig}
\usepackage{listings}

\usepackage[breaklinks]{hyperref}
\usepackage{microtype}

\def\BibTeX{{\rm B\kern-.05em{\sc i\kern-.025em b}\kern-.08em
    T\kern-.1667em\lower.7ex\hbox{E}\kern-.125emX}}

\tikzstyle{block} = [rectangle, draw, 
    text width=5em, text centered, rounded corners, minimum height=2em]
\tikzstyle{oval} = [ellipse, draw, 
    text width=5em, text centered, rounded corners, minimum height=2em]
\tikzstyle{bt} = [rectangle, draw, 
    text width=1em, text centered, rounded corners, minimum height=2em]
\usetikzlibrary{calc}
\usetikzlibrary{arrows.meta}
\usetikzlibrary{positioning}
\usetikzlibrary{fit}
\usetikzlibrary{shapes.geometric}

\colorlet{punct}{red!60!black}
\definecolor{background}{HTML}{EEEEEE}
\definecolor{delim}{RGB}{20,105,176}
\colorlet{numb}{magenta!60!black}


\lstdefinelanguage{json}{
    basicstyle=\normalfont\ttfamily,
    numbers=left,
    numberstyle=\scriptsize,
    stepnumber=1,
    numbersep=8pt,
    showstringspaces=false,
    breaklines=true,
    frame=lines,
    backgroundcolor=\color{background},
    literate=
     *{0}{{{\color{numb}0}}}{1}
      {1}{{{\color{numb}1}}}{1}
      {2}{{{\color{numb}2}}}{1}
      {3}{{{\color{numb}3}}}{1}
      {4}{{{\color{numb}4}}}{1}
      {5}{{{\color{numb}5}}}{1}
      {6}{{{\color{numb}6}}}{1}
      {7}{{{\color{numb}7}}}{1}
      {8}{{{\color{numb}8}}}{1}
      {9}{{{\color{numb}9}}}{1}
      {:}{{{\color{punct}{:}}}}{1}
      {,}{{{\color{punct}{,}}}}{1}
      {\{}{{{\color{delim}{\{}}}}{1}
      {\}}{{{\color{delim}{\}}}}}{1}
      {[}{{{\color{delim}{[}}}}{1}
      {]}{{{\color{delim}{]}}}}{1},
}

\begin{document}

\title{VizAPI: Visualizing Interactions between Java Libraries and Clients --- Artifact}

%\author{\IEEEauthorblockN{1\textsuperscript{st} Sruthi Venkatanarayanan}
%\IEEEauthorblockA{\textit{University of Waterloo} \\
%Waterloo, ON, CANADA \\
%s42venkat@uwaterloo.ca}
%\and
%\IEEEauthorblockN{2\textsuperscript{nd} Jens Dietrich
%\IEEEauthorblockA{\textit{Victoria University of Wellington}\\
%Wellington, NZ \\
%jens.dietrich@vuw.ac.nz}
%\and
%\IEEEauthorblockN{3\textsuperscript{rd} Craig Anslow}
%\IEEEauthorblockA{\textit{Victoria University of Wellington} \\
%Wellington, NZ\\
%craig.anslow@vuw.ac.nz}
%\and
%\IEEEauthorblockN{4\textsuperscript{th} Patrick Lam}
%\IEEEauthorblockA{\textit{University of Waterloo}\\
%Waterloo, ON, CANADA\\
%patrick.lam@uwaterloo.ca}
%}
%}

\author{\IEEEauthorblockN{Sruthi Venkatanarayanan\IEEEauthorrefmark{1}, Jens Dietrich\IEEEauthorrefmark{2}, Craig Anslow\IEEEauthorrefmark{2}, and Patrick Lam\IEEEauthorrefmark{1}}
\IEEEauthorblockA{
\begin{tabular}{cc}
\IEEEauthorrefmark{1}University of Waterloo & \IEEEauthorrefmark{2}Victoria University of Wellington\\
Waterloo, ON, Canada & Wellington, New Zealand \\
{\{s42venkat,patrick.lam\}@uwaterloo.ca}  & {\{jens.dietrich,craig.anslow\}@vuw.ac.nz}}
\end{tabular}
}
\maketitle

\begin{abstract}
The VizAPI tool shows visualizations that include both static and dynamic interactions between clients, the libraries they use, and  those libraries’ transitive dependencies (all written in Java). One application of VizAPI is by client developers, to answer a query about upstream code: will their code be affected by breaking changes in library APIs? Or, library developers can use VizAPI to find out about downstream code: which APIs in their source code are commonly used by clients? This document describes the VizAPI artifact.
\end{abstract}

\begin{IEEEkeywords}
static program analysis,
dynamic program analysis,
API usage,
software evolution,
software maintenance
\end{IEEEkeywords}


This document explains our artifact for the VizAPI tool, including how to obtain, install, and use the artifact. VizAPI generates a d3js visualization of library API usage by clients, using a modified version of Python's d3graph library. Our working Github repository for VizAPI is at \href{https://github.com/SruthiVenkat/api-visualization-tool}{https://github.com/SruthiVenkat/api-visualization-tool}. We have also archived the artifact at \href{https://doi.org/10.5281/zenodo.7023911}{https://doi.org/10.5281/zenodo.7023911} and our dataset at \href{https://doi.org/10.5281/zenodo.7023872}{https://doi.org/10.5281/zenodo.7023872}.

\section{Getting Started}

To acquire the repository, clone the repository (or download the archived version) and then copy the \texttt{apis-data} directory from the dataset into the root directory of the artifact. The API data directory contains our data for 101 benchmarks. If the input to VizAPI exists in \texttt{apis-data}, then VizAPI directly generates visualizations from the data files. If not, our package first runs the instrumentation to create the data files, and then generates visualizations from them.

We next describe the input to VizAPI. To run on a benchmark, the tool expects a JSON file called \texttt{input.json} as its input. The JSON file must contain an array of objects where each object describes a project; it can be in one of the following formats:

\begin{enumerate}
\item The first format, shown immediately below, is to be used when data for the project already exists in \texttt{apis-data}. In this case, the artifact ID of the project (which can be arbitrarily chosen by the user) and the type of the project (whether it is a client or library) need to be specified.
\lstinputlisting[language=json,label=listing:input-format-1]{input-format-1.json}
\item The second is to be used when data for the project does not exist in \texttt{apis-data}. Again, the artifact ID of the project and the type of the project (whether it is a client or library) still need to be specified. Since the data do not exist, VizAPI also needs to capture instrumentation data, and hence VizAPI expects the URL and commit ID of the project. VizAPI expects the project to be in Maven format and can automatically execute the project's tests.
\lstinputlisting[language=json,label=listing:input-format-2]{input-format-2.json}
\end{enumerate}

\section{Running VizAPI with Docker}
We recommend using Docker to run VizAPI; we have tested the configuration and believe that it should be portable.
\begin{enumerate}
\item Run \texttt{docker build -t img\_name .} 
from the directory where you have cloned the Github repository.
\item Run \texttt{docker run \\\hspace*{1em} -v /path/to/this/repo/api-\\\hspace*{1em} visualization-tool:/api-\\ \hspace*{1em} visualization-tool img\_name} 

The path before the : in the command is your local path to the repo. The path after the : in the command is the path in the container, which is \texttt{/api-visualization-tool}.
\end{enumerate}

\section{Running VizAPI without Docker}
It is also possible to run VizAPI outside of Docker.
\begin{enumerate}
\item Install the following Python packages: pandas, jupyterlab\_server, networkx, colourmap, python-louvain, sklearn, ismember, d3graph, PyGithub.
\item Change paths starting with \texttt{/api-visualization-\\tool} to point to the location of your repository in the following files: \texttt{api-viz.py}, \texttt{config/config.properties}.
\item Run \texttt{api-viz.py}.
\end{enumerate}

In both cases, i.e., with and without Docker, the final graph is generated with the name \texttt{api-usage.html}, in the VizAPI 
main directory.

The following is an example input.json needed to reproduce the Graph 1 at \href{https://sruthivenkat.github.io/VizAPI-graph/}{https://sruthivenkat.github.io/VizAPI-graph/}:
\lstinputlisting[language=json,label=listing:input-example]{input-example.json}

Some points to note are:
\begin{enumerate}
\item The size of the Docker image is around 4.1 GB.
\item The more data points, the longer the graphs take to generate.
\item When running VizAPI to generate graphs, you may see many Python Future warnings. They can be ignored.
\end{enumerate}

\end{document}
